Erinnerung: \lecture $ \omega \in \Omega^1(M) $ heißt geschlossen, falls $ \del{i}\omega_j . \del{j} \omega_i = 0 $ lokal in jeder Karte.\\
Idee: Definiere eine Abbildung $ d: \Omega^1 \to \Omega^2 $, die in Koordinaten gegeben ist als
\[ d\omega = d \big( \sum \omega_j dx^j \big) = \sum_{i < j} (\del{i} \omega_j - \del{j} \omega i) dx^i \wedge dx^j \]
Geschlossenheit wäre dann gleichbedeutend mit $d\omega = 0$.

\begin{defn}[äußere Ableitung, Cartan-Ableitung]\index{äußere Ableitung} \index{Cartan-Ableitung} \label{7.5}
	Sei $ V \subset \R^n $ offen. Sei $ \omega \in \Omega^k(V) $, also 
	\[ \omega = \sum_{i_1 < \dots < i_k} \omega_{i_1 \dots i_k} dx^{i_1} \wedge \dots \wedge dx^{i_k}. \]
	Dann definieren wir die \emph{äußere Ableitung} oder \emph{Cartan-Ableitung} $d\omega$ \emph{von $\omega$} vermöge
	\[ d\omega := \sum_{i_1 < \dots < i_k} \sum_j \del{j} \omega_{i_1 \dots i_k} dx^j \wedge dx^{i_1} \wedge \dots \wedge dx^{i_k}. \]
\end{defn}

\begin{rem*}
	\begin{enumerate}[label={\roman*})]
		\item $ \omega = f \in C^\infty(M) = \Omega^0(M) $\\
			$ df = \sum_j \del{j} f dx^j = df $ im Sinne der Definition
		\item $ \omega \in \Omega^1(M) $
			\begin{align*}
				d\omega &= \sum_j \sum_i \del{j}\omega_i dx^j \wedge dx^i = \sum_{j<i} \dotsc + \sum_{j > i} \dots\\
				&= \sum_{j<i} (\del{j} \omega_i - \del{i}\omega_j) dx^j \wedge dx^i
			\end{align*}
	\end{enumerate}
\end{rem*}

\begin{lem}
	Es gilt für $ V \subset \R^n, W \in \R^m $ offen
	\begin{enumerate}[label={\roman*})]
		\item $ d: \Omega^k(V) \to \Omega^{k+1}(V) $ linear
		\item $ d(\omega \wedge \eta) = d\omega \wedge \eta + (-1)^k \omega \wedge d\eta $ für $ \omega \in \Omega^k(V), \eta \in \Omega^l(V). $
		\item $d \circ d = 0$
		\item $ f: V \to W $ glatt, $\omega \in \Omega^k(W): f^*(d\omega) = d(f^*\omega)$
	\end{enumerate}
\end{lem}

\begin{thm}
	Ist $M$ eine differenzierbare Mannigfaltigkeit, so gibt es eine eindeutige lineare Abbildung $ d: \Omega^k(M) \to \Omega^{k+1}(M) $, sodass
	\begin{gather*}
		d(\omega \wedge \eta) = d\omega \wedge \eta + (-1)^k \omega \wedge d \eta \qquad \foralll \omega \in \Omega^k, \eta \in \Omega^l\\
		d \circ d = 0.
	\end{gather*}
	Für $ f \in \Omega^0 = C^\infty $ gilt:
	\begin{gather*}
		df \ \text{ist das Differential von $f$, also}\\
		df(X) = X(f) \qquad \foralll X \in \Xfrak^1(M)
	\end{gather*}
	In lokalen Koordinaten ist $d\omega$ gegeben wie in \ref{7.5}.
\end{thm}

\begin{lem}
	$ f: M \to N $ glatt mit differenzierbaren Mannigfaltigkeiten $M,N$. Dann gilt
	\[ f^*(d\omega) = d(f^*\omega) \qquad \foralll \omega \in \Omega^k(N). \]
\end{lem}

\begin{lem}[Koordinaten-unabhängige Definition von $d$]
	Sei $M$ eine differenzierbare Mannigfaltigkeit, $\omega \in \Omega^k(M), X_1, \dotsc, X_{k+1} \in \Xfrak(M)$. Dann ist
	\begin{align*}
		d\omega (X_1, \dots, X_{k+1}) =& \sum_{i=1}^{k+1} (-1)^{i-1} X_i \Big( \omega \big(X_1, \dotsc, \hat{X}_i, \dotsc, X_{k+1}\big) \Big) \\
			&+ \sum_{1 \leq i < j \leq k+1} (-1)^{i+j} \omega \big([X_i,X_j], X_1, \dotsc, \hat{X}_i, \dotsc, \hat{X}_j, \dotsc, X_{k+1} \big),
	\end{align*}
	wobei die Notation $\hat{X}_i$ bedeutet, dass dieser Faktor ausgelassen wird.
\end{lem}

\begin{rem*}
	Wegen $ [X_i,X_j] $ in der Formel genügt es nicht, $d\omega$ in $v_1, \dotsc, v_{k+1} \in T_pM$ auszuwerten, sondern man muss sie erst zu Vektorfeldern bei $p$ fortsetzen.
\end{rem*}

\begin{exmp}
	\begin{itemize}
		\item $ \omega \in \Omega^1(\R^3) $, also $ \omega = udx + vdy + wdz $
			\begin{align*}
				d\omega &= du \wedge dx + dv \wedge dy + dw \wedge dz\\
				\noalign{\centering\footnotesize $(du \wedge dx = \del{x}u dx + \del{y}u dy + \del{z}z dz) \wedge dx$}
				&= (\del{x}v - \del{y}u) dx \wedge dy + (\del{x}w - \del{z}u) dx \wedge dz + (\del{y}w - \del{z}v) dy \wedge dz
			\end{align*}
		\item $ \omega \in \Omega^1(\R^3) $, also $ \omega = u dx \wedge dy + v dx \wedge dz + w dy \wedge dz $\\
			$ d\omega = (\del{z}u - \del{y}v + \del{x}w) dx \wedge dy \wedge dz $
	\end{itemize}
\end{exmp}

\begin{rem}
	Vektor-Kalkül im $\R^3$\\
	\[ \grad f = \begin{pmatrix}
		\del{x}f \\ \del{y}f \\ \del{z}f
		\end{pmatrix} \quad \text{"Gradient"} \qquad (f \in C^\infty(\R^3)) \]
	\[ div X = \sum_{i=1}^3 \del{i} X^i \quad \text{"Divergenz"} \qquad (X \in \Xfrak^1(\R^3)) \]
	\begin{align*}
		rot X =& (\del{y}X^3 - \del{z}X^2) \del{x} \quad \text{"Rotation"} \qquad (X \in \Xfrak^1(\R^3))\\
		&+ (\del{z}X^1 - \del{x}X^3) \del{y}\\
		&+ (\del{x}X^2 - \del{y}X^1) \del{z}
	\end{align*}
	Betrachte nun $\bar{g}$, die Standard-Metrik auf $\R^3$.
	\begin{itemize}
		\item Sei $b$ die Abbildung
			\begin{align*}
				b&: \Xfrak(\R^3) \to \Omega^1(R^3)\\
				b(X) &= \hat{\bar{g}}(X) = \bar{g}(X,\cdot)\\
				&= \sum_{i,j} \bar{g}_{ij} X^j dx^i = \sum X_i dx^i
			\end{align*}
			$B$ ist eine Bündel-Isomorphismus.
		
		\item Sei $ \beta: \Xfrak(\R^3) \to \Omega^2(\R^3) $ die Abbildung
			\[ \beta(X):= X \lrcorner (dx \wedge dy \wedge dz). \]
			$\beta$ ist ein Bündel-Homomorphismus und sogar ein Isomorphismus: denn $\beta$ ist injektiv und da $\dim T_p\R^3 = 3 = \binom{3}{2} = \dim \Lambda^2 T_p\R^3$ auch surjektiv.
		
		\item Sei der "Hodge"-Stern die Abbildung
			\begin{align*}
				*: C^\infty(\R^3) &\to \Omega^3(\R^3)\\
				f &\mapsto f dx \wedge dy \wedge dz
			\end{align*}
			$*$ ist ein Bündel-Homomorphismus und sogar ein Isomorphismus, denn $*$ ist injektiv und $\dim \Lambda^0 T_p^*\R^3 = \binom{3}{0} = 1 = \binom{3}{3} = \dim \Lambda^3 T_p^*\R^3$. Dann kommutiert
			\[ \begin{tikzcd}
				C^\infty(\R^3) \arrow{r}{\grad} \arrow{d}{\id} & \Xfrak(\R^3) \arrow{r}{rot} \arrow{d}{b} & \Xfrak(\R^3) \arrow{r}{div} \arrow{d}{\beta} & C^\infty \arrow{d}{*}\\
				\Omega^0(\R^3) \arrow{r}{d} & \Omega^1(\R^3) \arrow{r}{d} & \Omega^2(\R^3) \arrow{r}{d} & \Omega^3(\R^3)
			\end{tikzcd} \]
			Wegen $ d \circ d = 0 $ folgt $rot \circ \grad = 0$.
	\end{itemize}
\end{rem}